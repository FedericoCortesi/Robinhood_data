\section{Assessing Risk Preferences: A Dual-Criterion Approach}
\subsection{Introduction}
The results of section \ref{return_section} paint very different pictures of retail investors. 
Using the method set forth by \cite{Welch2022} and \cite{Fedyk2024}, analysed also in section \ref{fedyk_paper_returns}, yields superior returns and similar drawdowns to the market.
They also conclude that the Robinhood crowd has achieved positve alpha when analysed under different factor models (table VII, IX, X in \cite{Welch2022} and table 16 in \cite{Fedyk2024}). 
These results appear to be in contrast with the existing literature on the retail investing, most notably \cite{BarberOdean2000}.

A more fundamental question, however, is whether those returns are attractive once investors' attitudes toward risk are taken into account.
This section evaluates the Robinhood portfolio against its market benchmarks using two complementary criteria.

First, I adopt the constant-relative-risk-aversion (CRRA) framework, in line with the majority of asset pricing work.
\begin{equation}
    U(W) = 
    \begin{cases}
    \frac{W^{1-\gamma}-1}{1-\gamma}, \gamma\neq 1\\
    \ln(W), \gamma = 1
    \end{cases}
\end{equation}

By computing expected utility of both the Robinhood and benchmark portfolios over a grid of possible risk aversion ($@g$) values,
I identify the cutoff $@g^*$ such that a representative CRRA investor is indifferent between the two.
This delivers a concise, parametric summary of how risk preferences may shape portfolio choice.

Second, I apply first- and second-order stochastic dominance tests (FSD and SSD) to the same return distributions.
This allows to avoid unnecesary, though conventional, assumptions regarding functional forms or parameterisation.
Log-Normality might well not be respected for different distributions, stochastic dominance tests take into account the shape of empirical CDFs to answer stronger questions. 

