\section{Introduction}

In recent years, the rise of commission-free trading platforms has profoundly reshaped retail investor behavior and sparked growing interest among scholars in behavioral finance. 
One of the most prominent examples is Robinhood, a mobile-first brokerage that gained widespread popularity for eliminating trading fees and offering a highly gamified user experience. 
The platform attracted a large number of retail investors, particularly young and inexperienced individuals\footnote{
According to Robinhood's IPO filing, the typical user on the platform is 31 years old, with an average account balance of approximately \$3,500. 
Notably, around half of the platform's users are investing for the first time.}.

An important development in the empirical literature was the release of Robintrack, an open-source dataset that tracks the number of Robinhood users holding individual stocks over time. 
This data, collected via Robinhood's public API, provides a rare opportunity to directly observe the trading dynamics and portfolio shifts of real retail investors. 
The dataset can be downloaded from \url{https://robintrack.net/}.

Several recent studies, including \cite{Fedyk2024} and \cite{Welch2022}\footnote{it must be noted that the former explicitly follows the method of the latter}, 
have leveraged the Robintrack dataset to examine retail investor performance. 
Their findings suggest that Robinhood investors, contrary to popular belief, exhibited strong market timing and outperformed passive benchmarks. 
In particular, these papers report significant cumulative returns and positive alpha using standard factor models.

In this paper, we revisit these claims by constructing an alternative methodology for portfolio formation based on the same dataset. 
Specifically, we analyze whether the returns of Robinhood users' favorite stocks exhibit stochastic dominance over benchmark indices.
Our goal is to offer a more nuanced assessment of whether retail investors truly generate abnormal returns or whether previous results may be driven by sample selection or methodological choices.