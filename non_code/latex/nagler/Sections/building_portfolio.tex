\section{Building the Robinhood Portfolio}
\subsection{Methodology}
\subsubsection{Weights Methods} 
\cite{Fedyk2024} and \cite{Welch2022} use the same approach to build the performance of the Robinhood crowd (or "reference index"): 
they build daily weights and then apply the weights from the previous day to daily stock returns, directly building portfolio returns. 

First, it is necessary to define how those weights are computed. They define two different types of weights, although they yield similar findings in their analysis.

The first method is the "dollar method", which assumes that every investor represents an equal dollar amount investment in the stock. 
\begin{equation}
    w^{\text{dollar}}_{i,t} = \frac{N_{i,t}}{\sum_j N_{j,t}}
\end{equation}

where $w_{i,t}$ is the Robinhood portfolio weight of security $i$ at time $t$ and $N_{i,t}$ is the number of investors in security $i$ at time $t$.

Alternatively, they define the "share method", where each Robinhood investor in a stock represents a one share investment in that stock.
\begin{equation}
    w^{\text{share}}_{i,t} = \frac{N_{i,t}\cdot P_{i,t}}{\sum_j N_{j,t}\cdot P_{j,t}}
\end{equation}
where $P_{j,t}$ is the price of stock $j$ at time $t$.

\subsubsection{Different Approaches to Compute Performance}
As explained above, the biggest limitation of the Robintrack dataset is that it counts the number of users holding a certain security and doesn't provide any information on the amount invested in a particular security. 

The other authors build the portfolio returns by multiplying weights by their daily returns\footnote{returns are computed directly by CRSP and are adjusted for dividends, e.g. if $P_0=10$ and $D_1=5$ and $P_1=5$ returns would be 0\%}, 
assuming that the weights, however computed, represent a certain share of wealth in a stock of the Robinhood crowd.
\begin{equation}
    r_{RH,t} = \sum_{i=1}^N w_{i,t}\cdot r_{i,t}
    \label{returns_fedyk}
\end{equation}

On the other hand, I tried to compute the value of the Robinhood portfolio by doing a weighted sum of the prices of the securities in the dataset.
Conceptually, this represents the portfolio of an investor who decides to allocate a certain number (or percentage) of shares to each security.
I will call this the "Price" method, or simply my method.

We can therefore define the value of the Robinhood Portfolio as follows:
\begin{equation}
    V_{RH,t}=\sum_{i=1}^N w^{\text{dollar}}_{i,t}\cdot P_{i,t}
    \label{value_mine}
\end{equation}

Returns are then computed from the value of the overall portfolio as:
\begin{equation}
    r_{RH,t} = \ln\left(\frac{V_{RH,t}}{V_{RH,t-1}}\right)
\label{returns_mine}
\end{equation}

In this paper I prefer using log-returns where possible. 

\subsubsection{Capturing the Persistence of Investor Composition}
Although both approaches ultimately yield a time series of Robinhood portfolio returns, there is a fundamental difference in what these return paths represent.

In the method used by \cite{Fedyk2024} and \cite{Welch2022}, the portfolio is effectively rebalanced every day to reflect the current composition of investor popularity.
Each day's return is computed based on that day's weights and the corresponding daily stock-level returns.
This provides a valid snapshot of the average return generated by the stocks held on a given day.

However, this approach does not preserve the economic exposure that investors accumulate through time.
A stock that was extremely popular for several days but declines in popularity just before a price spike will have minimal influence on the portfolio's return when that spike occurs.
Only the weights at time $t-1$ affect the return at time $t$\footnote{Previous day's weights are taken to prevent look-ahead bias}, so the model captures immediate sentiment shifts but not the cumulative effects of holding positions over time.

In contrast, the methodology I propose (\ref{returns_mine}) applies weights to stock prices and computes returns from changes in total portfolio value.
This implies that a stock that was heavily weighted yesterday continues to influence portfolio performance today, even if its popularity has declined.
The return reflects both the dynamics of price changes and the path dependency of investor composition.

As a result, my method embeds the effects of investor flows, popularity shifts, and concentration in the actual evolution of portfolio value.
The cumulative performance is not a sequence of disconnected daily snapshots, but a reflection of how crowd behavior builds, persists, and unwinds over time.

Conceptually, this distinction is important when studying behavioral dynamics.
Retail investor behavior—particularly on platforms like Robinhood—is driven not only by cross-sectional preferences at a point in time but also by persistent patterns of attention, sentiment, and herding.
A portfolio that evolves with these behavioral shifts provides a more realistic measure of the actual wealth path experienced by retail investors, rather than an idealized, continually rebalanced index.

In this sense, computing returns from the portfolio value offers a more structurally consistent and behaviorally meaningful representation of the Robinhood crowd's investment trajectory.