\section{Literature Review}

The rise of commission-free trading platforms such as Robinhood has significantly altered the behavior of retail investors, leading to increased speculative trading and attention-driven investment patterns. Existing literature highlights the unique characteristics of these investors, the role of fintech innovations in shaping trading behavior, and the broader market impact of retail-driven trading.

\subsection{Retail Investors and Attention-Induced Trading}
One of the defining characteristics of Robinhood users is their tendency to engage in attention-induced trading. \cite{barber2011individual} document that retail investors are heavily influenced by limited attention and past return performance, often buying stocks that experience large daily price movements. This behavioral bias is further reinforced by Robinhood's design, which prominently features "Top Movers" lists and real-time price changes \cite{barber2021robinhood}. 

Ardia et al. \cite{Ardia2023Fast} provide a high-frequency analysis of Robinhood traders, showing that they react aggressively within an hour of extreme negative price movements. This suggests that Robinhood users are particularly prone to rapid contrarian trading, potentially due to real-time engagement with financial markets through mobile notifications and social media.

\subsection{The Influence of Robinhood on Retail Trading Behavior}
The trading behavior of Robinhood users differs from that of traditional retail investors. The simplicity and gamification of the app influence decision-making, leading to increased speculative trading. \cite{barber2021robinhood} find that Robinhood's "Top Mover" feature encourages users to buy both extreme gainers and losers at higher rates than traditional retail investors, diverging from the broader retail trading pattern of favoring winners.

Moreover, Robinhood traders exhibit a strong preference for stocks experiencing extreme price movements. \cite{Ardia2023Fast} report that users react particularly fast to negative price shocks, displaying contrarian tendencies not typically observed in traditional retail trading.

\subsection{Market Impact of Robinhood Users}
Several studies have examined the broader market impact of Robinhood-driven trading, particularly during periods of heightened retail activity. Concentrated retail buying can lead to short-term price pressure and subsequent reversals. \cite{barber2021robinhood} show that Robinhood herding episodes-where a large number of users buy the same stock-result in negative abnormal returns of approximately 4.7\% over the following 20 days. 

Additionally, retail-driven trading was amplified during the COVID-19 pandemic, with a surge in new users engaging in speculative trading. \cite{Ardia2023Fast} find that Robinhood users intensified their trading activity post-pandemic announcement, particularly in small-cap and volatile stocks. 


