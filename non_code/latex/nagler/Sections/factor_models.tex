\section{Factor Model Evaluation and Alpha Robustness}
In this section I subject the representative Robinhood portfolio to a series of standard asset pricing tests in order to challenge the unusually large abnormal returns reported in \cite{Fedyk} and \cite{Welch2022}, 
more precisely between 14.8\% and 21.4\% annualized alpha.  
I estimate the Capital Asset Pricing Model, the Fama-French three- and four-factor specifications, and the six-factor model using exactly the same factor premia from Kenneth French's database that Fedyk and Welch employs.  
Unlike his analysis, which is confined to a hand-picked subset of stocks, I apply the portfolio construction procedure developed earlier in this paper to the full universe of available securities.  
I also extend the sample window through the COVID-19 crash to test the stability of the alpha estimates under extreme market stress.  
The resulting intercepts are far more muted than those in Fedyk's tables, and many lose statistical significance once pandemic volatility is incorporated.  
These results call into question the realism of the super-high alpha values in the existing literature and highlight the importance of broad coverage and rigorous methodology in evaluating retail-driven strategies.  
